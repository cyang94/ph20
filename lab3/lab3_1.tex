
\documentclass[english]{article}
\usepackage[T1]{fontenc}
\usepackage[latin9]{inputenc}
\usepackage{geometry}
\geometry{verbose,tmargin=2cm,bmargin=2cm,lmargin=2cm,rmargin=2cm}
\setlength{\parskip}{\smallskipamount}
\setlength{\parindent}{0pt}
\usepackage{float}
\usepackage{graphicx}

\makeatletter

%%%%%%%%%%%%%%%%%%%%%%%%%%%%%% LyX specific LaTeX commands.
%% A simple dot to overcome graphicx limitations
\newcommand{\lyxdot}{.}


\makeatother

\usepackage{babel}
\begin{document}
\begin{center}
\textbf{\Large{Ph 20.2 - Introduction to Numerical Techniques: Numerical
Integration}}
\par\end{center}{\Large \par}

\begin{center}
{\large{Cassidy Yang\bigskip{}
}}
\par\end{center}{\large \par}


\subsection*{Part I:}


\paragraph{1. }

The equation for the mass on a spring is:

\begin{equation}
F=ma=-kx.\label{eq:1}
\end{equation}


Additionally, we applied $v=\frac{dx}{dt}$, and $a=\frac{dv}{dt}=-\frac{kx}{m}$,
and used the Euler method to approximate $x$ and $v$ as a function
of time. Figure 1 plots the approximated position (1a) and velocity
(1b) with respect to time ($h=0.01$, $x_{0}=0$, $v_{0}=1$, $\frac{k}{m}=1$).
Note that after a few oscillations, the amplitudes obviously increase,
indicating a divergence from the exact solution, and we will se later
that this indicates that the approximated energy increases over time.

\begin{figure}[H]


\caption{Approximate Position and Velocity vs. Time}


\begin{minipage}[t]{0.5\columnwidth}%
\noindent \begin{center}
\includegraphics[scale=0.3]{C:/Users/cyang94/Dropbox/Ph20/lab3/Position}
\par\end{center}

\noindent \begin{center}
(a)
\par\end{center}%
\end{minipage}%
\begin{minipage}[t]{0.5\columnwidth}%
\noindent \begin{center}
\includegraphics[scale=0.3]{C:/Users/cyang94/Dropbox/Ph20/lab3/Velocity}
\par\end{center}

\noindent \begin{center}
(b)
\par\end{center}%
\end{minipage}

\end{figure}



\paragraph{2.}

We again used the same conditions as above, and plotted the global
errors ($x_{anal}(t_{i})-x_{i}$ and $v_{anal}(t_{i})-v_{i}$ ) for
position (blue) and velocity (green) in Figure 2.

\begin{figure}[H]
\caption{Errors in Position and Velocity vs. Time}


\noindent \centering{}\includegraphics[scale=0.5]{C:/Users/cyang94/Dropbox/Ph20/lab3/Errors}
\end{figure}



\paragraph{3.}

We determined the maximum error in position for particular step sizes
beginning with $h_{0}=0.4$ (Figure 3). Note that for small step sizes
$(\sim h<0.1)$, the maximum error position seems to decrease linearly.

\begin{figure}[H]
\caption{Maximum $x$ Error vs. Step Size}


\noindent \centering{}\includegraphics[scale=0.5]{C:/Users/cyang94/Dropbox/Ph20/lab3/ErrorvsStepSize}
\end{figure}



\paragraph{4.}

The normalized total energy of the system can be calculated using
$E=x^{2}+v^{2}$. We overlayed the evolution of energy with the evolution
of error in Figure 4. The absolute error of the position is shown
in blue. The normalized total energy of the system is shown in green.
The error in energy is shown in red (the energy at all times should
be equal to one in the exact solution). Note that the error in energy
increases much faster than that of position.

\begin{figure}[H]


\caption{Evolution of Energy and Error over Time}


\noindent \centering{}\includegraphics[scale=0.5]{C:/Users/cyang94/Dropbox/Ph20/lab3/EandErrors}
\end{figure}



\paragraph{5.}

We have the system (assuming that we set $\frac{k}{x}=1$):

\begin{equation}
x_{i+1}=x_{i}+hv_{i+1},\,\,\, v_{i+1}=v_{i}-hx_{i+1}.\label{eq:2}
\end{equation}


In order to solve it, we can substitute the second equation into the
first:

\[
x_{i+1}=x_{i}+h(v_{i}-hx_{i+1})
\]


\begin{equation}
x_{i+1}=\frac{x_{i}+hv_{i}}{h^{2}+1}\label{eq:3}
\end{equation}


\begin{equation}
v_{i+1}=v_{i}-hx_{i+1}=v_{i}-h\left(\frac{x_{i}+hv_{i}}{h^{2}+1}\right).\label{eq:4}
\end{equation}


The position and velocities are shown in Figure 5 a and b respectively.
Note that the amplitudes decrease over time, indicating that the approximated
energy decreases over time. Additionally, the maximum error in position
with respect to step size (for the same step sizes as before) is shown
in Figure 6. Again, the global errors seem to decrease linearly after
a certain point. The evolution of approximate energy is shown in Figure
7. Again, the blue curve indicates the absolute error of position;
the green curve indicates the approximate energy; and the red curve
indicates the error of the energy. Note that as opposed to the explicit
method, the implicit method's energy decreases over time.

\begin{figure}[H]
\caption{Implicit Approximate Position and Velocity vs. Time}


\begin{minipage}[t]{0.5\columnwidth}%
\noindent \begin{center}
\includegraphics[scale=0.3]{C:/Users/cyang94/Dropbox/Ph20/lab3/P_Im}
\par\end{center}

\noindent \begin{center}
(a)
\par\end{center}%
\end{minipage}%
\begin{minipage}[t]{0.5\columnwidth}%
\noindent \begin{center}
\includegraphics[scale=0.3]{C:/Users/cyang94/Dropbox/Ph20/lab3/V_Im}
\par\end{center}

\noindent \begin{center}
(b)
\par\end{center}%
\end{minipage}
\end{figure}


\begin{figure}[H]


\caption{Maximum $x$ Error and Step Size (Implicit Euler Method)}


\noindent \centering{}\includegraphics[scale=0.5]{C:/Users/cyang94/Dropbox/Ph20/lab3/ErrorvsStepSize}
\end{figure}


\begin{figure}[H]
\caption{Evolution of Energy and Error (Implicit Euler Method)}


\noindent \centering{}\includegraphics[scale=0.5]{C:/Users/cyang94/Dropbox/Ph20/lab3/EandErrors}
\end{figure}



\subsection*{\medskip{}
}


\subsection*{Part II:}


\paragraph{1. }

The phase space geometry of the implicit and explicit methods are
shown in Figure 8. The explicit method trajectory is shown in blue;
the implicit method trajectory is shown in green; and the trajectory
of the exact solution is shown in red.

\begin{figure}[H]
\caption{Phase-Space Geometry (Implicit and Explicit Methods)}


\noindent \centering{}\includegraphics[scale=0.5]{C:/Users/cyang94/Dropbox/Ph20/lab3/Phase_Ex_Im}
\end{figure}



\paragraph*{2.}

The phase-space trajectory generated from the symplectic method is
shown in Figure 9 with the exact trajectory in green and the approximated
trajectory in blue. Figure 9a has the same step size as above. Figure
9b uses a larger step size ($h=0.1$) so that it becomes more apparent
that although the trajectory is closed (indicating constant energy),
it still deviates slightly from that of the exact solution by some
phase.

\begin{figure}[H]
\caption{Phase-Space Geometry (Symplectic)}


\noindent %
\begin{minipage}[t]{0.5\columnwidth}%
\noindent \begin{center}
\includegraphics[scale=0.3]{C:/Users/cyang94/Dropbox/Ph20/lab3/Phase_Sym1}
\par\end{center}

\noindent \begin{center}
(a)
\par\end{center}%
\end{minipage}%
\begin{minipage}[t]{0.5\columnwidth}%
\noindent \begin{center}
\includegraphics[scale=0.3]{C:/Users/cyang94/Dropbox/Ph20/lab3/Phase_Sym}
\par\end{center}

\noindent \begin{center}
(b)
\par\end{center}%
\end{minipage}
\end{figure}



\paragraph*{3.}

The evolution of energy obtained with the symplectic Euler method
is shown in Figure 10. Note that the energy oscillates within the
vincinity of 1 and does not deviate from that value even after many
oscillations.

\begin{figure}[H]


\caption{Evolution of Energy (Symplectic Euler Method)}


\noindent \begin{centering}
\includegraphics[scale=0.5]{C:/Users/cyang94/Dropbox/Ph20/lab3/Energy_Sym}
\par\end{centering}

\end{figure}



\paragraph*{4.}

After many oscillations, the phase lag of our approximate values become
visible (Figure 11).

\begin{figure}[H]


\caption{Phase Lag of Symplectic Euler Method}


\noindent \begin{centering}
\includegraphics[scale=0.5]{C:/Users/cyang94/Dropbox/Ph20/lab3/OscillationLag}
\par\end{centering}

\end{figure}

\end{document}
